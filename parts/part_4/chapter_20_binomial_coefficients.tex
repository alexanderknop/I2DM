\chapter{Binomial Coefficients}
\label{chapter:binomials}
\marginurl{%
  Permutations and Binomial Coefficients:\\\noindent
  Introduction to Combinatorics \#4
}{youtu.be/HLClazoqqzg}
This chapter studies the following question:
``how many ways to take $k$ objects out of a box with $n$ objects''.
We assume that the objects are taken one by one; note that there are four modes
for this question.
\begin{enumerate}
  \item we return objects to the box after we take them and the order in which
    we take them matters,
  \item we are \emph{do not} return the objects and the order in which
    we take them matters,
  \item we return objects to the box after we take them and the order in which
    we take them \emph{does not} matter,
  \item we are \emph{do not} return the objects and the order in which
    we take them \emph{does not} matter.
\end{enumerate}

The Table~\ref{table:selections} summarizes the results we are going
to prove.
\begin{table}[h!]
  \centering
  \begin{tabular}{lll}
    \toprule
    Object's name & Parameters & Formula \\
    \midrule
    \multirow{2}{*}{Functions} & we return objects and & \multirow{2}{*}{$n^k$} \\
                               & the order \emph{is} important & \\
    \multirow{2}{*}{Injections} & we \emph{do not} return objects and &
      \multirow{2}{*}{$\numberOfPermutations[k]{n}$} \\
                                & the order \emph{is} important & \\
    \multirow{2}{*}{Subsets} & we \emph{do not} return objects and & \multirow{2}{*}{$\binom{n}{k}$} \\
                             & the order \emph{is not} important & \\
    \multirow{2}{*}{Multisets} & we return objects and & \multirow{2}{*}{$\binom{n + k - 1}{k}$} \\
                               & the order \emph{is not} important & \\
    \bottomrule
  \end{tabular}
  \caption{Formulas for the numbers of ways to take $k$ objects out of a box
  with $n$ objects}
  \label{table:selections}
\end{table}

\section{Counting Functions}
Note that if we number objects using numbers from $1$ to $n$, then in the first
mode the answer is the same as the number of functions from $[n]$ to $[k]$ since
we need to just choose which objects is selected on the $i$th step for
$i \in \range{k}$.

Let us solve a more general question; assume we have two finite sets $X$ and
$Y$: how many functions exist from $X$ to $Y$?


\begin{theorem}
\label{theorem:number-of-functions}
  Let $X$ and $Y$ be some finite sets. $\functions{X}{Y}$ represents the set of all
  functions from $X$ to $Y$. Then $\cardinality{\functions{X}{Y}} =
  \cardinality{Y}^{\cardinality{X}}$.
\end{theorem}
\nomenclature[S]{$\functions{A}{B}$}{denotes the set of functions from $A$ to $B$}
\begin{proof}
  For simplicity we prove the statement in the case when $X = \range{n}$. Fix some
  finite set $Y$. We prove the statement using induction by $n$. The base case
  for $n = 1$ is obvious, since there are $\cardinality{Y}$ different functions
  from $\range{1}$ to $Y$. Let us prove the induction step, by the induction
  hypothesis, $\cardinality{\functions{\range{n - 1}}{Y}} = 
  \cardinality{Y}^{n - 1}$. Note that
  \begin{multline*}
    \cardinality{\functions{\range{n}}{Y}} = 
      \cardinality{\set[f \in Y^{[ n - 1]}, y \in Y]{(f, y)}} = \\
    \cardinality{\functions{\range{n - 1}}{Y} \times Y} = 
      \cardinality{\functions{\range{n - 1}}{Y}} \cdot \cardinality{Y} =
      \cardinality{Y}^n.
  \end{multline*}
\end{proof}

\begin{corollary}
  There are $n^k$ ways to select $k$ objects out of $n$ if the order matters
  and we return objects to the box after we pick them.
\end{corollary}

\begin{exercise}
  Finish the proof of Theorem~\ref{theorem:number-of-functions} by proving that
  the statement holds for any set $X$.
\end{exercise}

However, what if we need to find size of a subset of $Y^X$ satisfying some
constraint? For example, we may try to find the size of the set
\[
  \injections{X}{Y} = \set[f\text{ is an injection}]{f \in \functions{X}{Y}}.
\]
\nomenclature[S]{$\injections{A}{B}$}{denotes the set of injections from $A$ to $B$}
First, let us try to do this informally. Assume that $X = \range{n}$ and
$\cardinality{Y} = m$,
to define $f \in \injections{X}{Y}$ we need to choose images of $1$, $2$, \dots,
$n$. There are $m$ possible ways to select an image of $1$, $m - 1$ ways to
define $f(2)$ since we cannot use the value selected for $1$ etc. Hence,
$\cardinality{\injections{X}{Y}} = m (m - 1) \dots (m - n + 1)$ (we denote this
number as $\numberOfPermutations[n]{m}$\footnote{%
  In combinatorics the notation $\numberOfPermutations[n]{m}$ is widely used and
  is also called the Pochhammer symbol. However, Knuth's $m^{\underline{n}}$
  notation is increasingly popular. It is also important to note that sometimes 
  $\numberOfPermutations[n]{m}$ denotes $m (m + 1) \dots (m + n - 1)$. Moreover,
  Leo August Pochhammer, the author of the Pochhammer symbol, used the
  Pochhammer symbol to denote the binomial coefficients.
}).
\nomenclature[C]{$\numberOfPermutations[n]{m}$}{denotes the number of ways to choose a
subset of $n$ elements from a fixed set of $m$ elements}

\begin{theorem}
\label{theorem:number-of-injections}
  Let $X$ and $Y$ be some finite sets. Then $\cardinality{\injections{X}{Y}} =
  \numberOfPermutations[\cardinality{X}]{\cardinality{Y}}$.
\end{theorem}
\begin{proof}
  Let us prove this statement for $X = \range{n}$. We prove this using induction by
  $n$. The base case, for $n = 1$, is clear. Now we need to prove the induction
  step from $n$ to $n + 1$. By the induction hypothesis, for any $m$, the
  number of injections from $\range{n}$ to $Y$ is equal to
  $\numberOfPermutations[n]{\cardinality{Y}}$.

  Fix some $m$ and some set $Y$ of cardinality $m$. Note that
  \[
    \cardinality{\injections{X}{Y}} =
    \cardinality{
      \set[v \not\in \Im f]{(f, v) \in \injections{\range{n - 1}}{Y} \times \range{m}}
    }.
  \]
  It is easy to see that $\cardinality{\set[v \not\in \Im f]{(f, v)}} = m - n + 1$
  for any $f \in \injections{\range{n - 1}}{Y}$ and
  \[
    \set[v \not\in \Im f]{(f, v) \in \injections{\range{n - 1}}{Y} \times
      \range{m}} =
    \bigcup_{f \in \injections{\range{n - 1}}{Y}} \set[v \not\in \Im f]{(f, v)}.
  \]
  As a result, $\cardinality{\injections{X}{Y}} = 
    \numberOfPermutations[n - 1]{m} \cdot (m - n + 1) =
    \numberOfPermutations[n]{m}$.
\end{proof}

The special case of this result is that there are
$n \cdot (n - 1) \cdot  \dots \cdot  1$ different
permutations of $\range{n}$ (recall that the number is denoted by
$\numberOfPermutations{n}$).

\begin{exercise}
  Finish the proof of Theorem~\ref{theorem:number-of-injections} by proving that
  the statement holds for any finite set $X$.
\end{exercise}

\begin{corollary}
  There are $\numberOfPermutations[k]{n}$ ways to select $k$ objects out of $n$ if the order matters
  and we do not return objects to the box after we pick them.
\end{corollary}

\section{Counting Subsets}
In this section we study the version of the question when we do not return the
objects back to the box; i.e., we cannot select an object twice.

Recall that we denoted the set of all subsets of $X$ by $2^X$. The reason for
this notation is that $\cardinality{\subsets{X}} = 2^{\cardinality{X}}$. A quite
famous example of a subset of this set is the set
\[
  \subsets[n]{X} = \set[\cardinality{A} = n]{A \subseteq X}.
\]
\nomenclature[S]{$\subsets[k]{A}$}{denotes the set of subsets of $A$ of
cardinality $k$}
In other words, it is the set of all possible ways to select $n$ elements from
$X$. Size of the set $\subsets[n]{\range{m}}$ we denote by $\binom{m}{n}$ and
call it a binomial coefficient.
\nomenclature[C]{$\binom{m}{n}$}{denotes the number of ways to choose an
unordered subset of $n$ elements from a fixed set of $m$ elements}
\begin{exercise}
  Show that for any two finite sets $X$ and $Y$, if $\cardinality{X} =
  \cardinality{Y}$, then
  $\cardinality{\subsets[k]{X}} = \cardinality{\subsets[k]{Y}}$.
\end{exercise}

Note that by any ordered selection of $n$ object out of $m$, one may construct
an unordered selection of $n$ objects out of $m$, and each unordered selection
is counted $n!$.
\begin{theorem}
\label{theorem:binomial-coefficients-explicit}
  For any $n > k \ge 0$,
  $\binom{n}{k} = \frac{\numberOfPermutations[k]{n}}{k!} = \frac{n!}{k!  (n - k)!}$.
\end{theorem}

\begin{exercise}
  Show that $\binom{n}{k} = \binom{n}{n - k}$ for any $n > k$.
\end{exercise}

The formula in the Theorem~\ref{theorem:binomial-coefficients-explicit} allows
to find the values of binomial coefficients, however, it is not very convenient
since $n!$ is growing very fast. Thus the following theorem provides a much more
efficient way to compute the values of binomial coefficients.

\begin{theorem}[Pascal's rule]
\label{theorem:pascals-rule}
  For $n > k \ge 1$, $\binom{n}{k} = \binom{n - 1}{k - 1} + \binom{n - 1}{k}$.
\end{theorem}
\begin{proof}
  The first, algebraic, proof of this theorem is quite simple, we just notice
  that
  \begin{multline*}
    \binom{n - 1}{k - 1} + \binom{n - 1}{k} =
      \frac{(n - 1)!}{(k - 1)!(n - k)!} + \frac{(n - 1)!}{k!(n - k - 1)!} = \\
      \frac{(n - 1)!}{(k - 1)! (n - k - 1)!}
        \left(
          \frac{1}{n - k} + \frac{1}{k}
        \right) = \frac{n!}{k! (n - k)!} = \binom{n}{k}.
  \end{multline*}

  However, this proof does not explain \emph{why} the statement is true.
  So we consider an alternative proof, which informally can be explained as
  follows. Assume we need to choose $k$ objects out of $n$. There are two
  possible ways:
  \begin{itemize}
    \item we may select $n$ and choose $k - 1$ objects from the rest,
    \item or we may decide to not select $n$ choose $k$ objects from the rest.
  \end{itemize}
  In the first case we have $\binom{n - 1}{k - 1}$ ways to select objects and
  in the second case we have $\binom{n - 1}{k}$ ways to select objects.

  Let us prove the statement a bit more formally. Note that
  \begin{multline*}
    \subsets[k]{\range{n}} = 
      \set[\cardinality{A} = k \text{ and } n \in A]{A \subseteq \range{n}} \cup \\
    \set[\cardinality{A} = k \text{ and } n \notin A]{A \subseteq \range{n}}.
  \end{multline*}
  Since these sets are disjoint and 
  $\set[\cardinality{A} = k \text{ and } n \notin A]{A \subseteq \range{n}} = 
  \subsets[k]{\range{n - 1]}}$, we get the following equality
  \[
    \binom{n}{k} = 
    \cardinality{\set[\cardinality{A} = k \text{ and } n \in A]{A \subseteq \range{n}}} +
    \binom{n - 1}{k}.
  \]

  Hence, to finish the proof we need to explain that
  \[
    \cardinality{\set[\cardinality{A} = k \text{ and } n \in A]{A \subseteq \range{n}}} =
    \binom{n - 1}{k - 1}.
  \]
  To prove this statement we construct a bijection
  \[
    f : \set[\cardinality{A} = k \text{ and } n \in A]{A \subseteq \range{n}} \to
      \subsets[k]{\range{n - 1}}
  \]
  such that $f(A) = A \setminus \set{n}$.
  It is clear that this is a bijection. Thus, we prove the statement.
\end{proof}

A mnemonic rule for the Pascal's rule is to use Pascal's triangle.
\footnote[][-8cm]{
  The pattern of numbers that forms Pascal's triangle was known well before
  Pascal's time.  Halayudha, around 975 explained obscure references to
  Meru-prastaara, the Staircase of Mount Meru, giving the first surviving
  description of the arrangement of these numbers into a triangle.

  The Persian mathematician Al-Karaji (953–1029) wrote a now lost book which
  contained the first description of Pascal's triangle. It was later repeated by
  the Persian poet-astronomer-mathematician Omar Khayyám (1048–1131); thus the
  triangle is also referred to as the Khayyam triangle in Iran.

  Pascal's triangle was known in China in the early 11th century through the
  work of the Chinese mathematician Jia Xian (1010–1070). In the 13th century,
  Yang Hui (1238–1298) presented the triangle and hence it is still called Yang
  Hui's triangle in China.

  Pascal's Traité du triangle arithmétique (Treatise on Arithmetical Triangle)
  was published in 1655. In this, Pascal collected several results then known
  about the triangle, and employed them to solve problems in probability theory.
  The triangle was later named after Pascal by Pierre Raymond de Montmort (1708)
  who called it ``Table de M. Pascal pour les combinaisons'' (French: Table of
  Mr. Pascal for combinations) and Abraham de Moivre (1730) who called it
  ``Triangulum Arithmeticum PASCALIANUM'' (Latin: Pascal's Arithmetic Triangle),
  which became the modern Western name.
}
\begin{figure}
  \centering
  \begin{tabular}{lccccccccc}
    &    &    &    &  1\\\noalign{\smallskip\smallskip}
    &    &    &  1 &    &  1\\\noalign{\smallskip\smallskip}
    &    &  1 &    &  2 &    &  1\\\noalign{\smallskip\smallskip}
    &  1 &    &  3 &    &  3 &    &  1\\\noalign{\smallskip\smallskip}
    1 &    &  4 &    &  6 &    &  4 &    &  1\\\noalign{\smallskip\smallskip}
  \end{tabular}
\end{figure}
In this diagram the $k$th entry of the $n$th row
(entries and rows have numbers starting from $0$) is equal to $\binom{n}{k}$.
Thus the rule for the triangle is very simple, the value of an entry is equal
to $1$ if it is the first or the last in the row or it is equal to the sum
of the two entries to the left and right on the row above.

\begin{exercise}
  Show that $\binom{n}{k} = \binom{n}{n - k}$ for any integers $n > k \ge 0$
\end{exercise}


\subsection{Binomial Theorem}
Now we are ready to prove the theorem which gave the name to binomial
coefficients.
\begin{theorem}[Binomial theorem]
\label{theorem:binomial}
  For any real numbers $x$ and $y$,
    \[
      \sum_{k = 0}^n \binom{n}{k} x^k y^{n - k} = (x + y)^n.
    \]
\end{theorem}
\begin{proof}
  Informally, the explanation of the equality is as follows.
  If we consider the product
  \[
    \underbrace{(x + y) \cdot (x + y) \cdot
      \ldots \cdot (x + y)}_{n \text{ times}},
  \]
  then for every $k$ there are exactly $\binom{n}{k}$ possibilities to obtain
  $x^k y^{n - k}$. Indeed, to obtain $x^k y^{n - k}$ we need to choose $x$ from
  $n$ possibilities (corresponding to the multiplier $x + y$) exactly $k$ times.

  A formal proof uses the induction by $n$. The base case is
  true, since $\sum_{k = 0}^1 \binom{1}{k} x^k y^{1 - k} = x + y =
  (x + y)^1$. Assume that
  \[
    \sum_{k = 0}^n \binom{n}{k} x^k y^{n - k} = (x + y)^n,
  \]
  we wish to prove that
  \[
    \sum_{k = 0}^{n + 1} \binom{n + 1}{k} x^k y^{n + 1 - k} =
      (x + y)^{n + 1}.
  \]
  Note that
  \begin{multline*}
    (x + y)^{n + 1} = (x + y)
      \left(
        \sum_{k = 0}^n \binom{n}{k} x^k y^{n - k}
      \right) = \\
    \sum_{k = 0}^n \binom{n}{k} x^{k + 1} y^{n - k} +
      \sum_{k = 0}^n \binom{n}{k} x^{k} y^{n + 1 - k} = \\
    \sum_{k = 1}^{n + 1} \binom{n}{k - 1} x^k y^{n + 1 - k} +
      \sum_{k = 0}^n \binom{n}{k} x^k y^{n + 1 - k} = \\
    \sum_{k = 0}^{n + 1}
      \left(
        \binom{n}{k - 1} + \binom{n}{k}
      \right)
      x^k y^{n + 1 - k} =
    \sum_{k = 0}^{n + 1} \binom{n + 1}{k} x^k y^{n + 1 - k}.
  \end{multline*}
\end{proof}

Finally, we need to answer the question in the mode, when the order does not
matter and we do not return the objects to the box. The answer to this question
is clearly equal to the number of multisets of $\range{n}$ containing $k$
objects.
\begin{theorem}
\label{theorem:multisets}
  The number of $k$-element multisets whose elements all belong to $\range{n}$ is
  $\binom{n + k - 1}{k}$.
\end{theorem}

\begin{exercise}
  Prove Theorem~\ref{theorem:multisets}
\end{exercise}

Using \Cref{theorem:binomial} one may prove several important equalities of sums
of binomial coefficients.
\begin{corollary}
\label{corollary:binommials-equality}
  Let $n \in \N$. Then
  \begin{enumerate}
    \item $\sum_{k = 0}^n (-1)^k \binom{n}{k} = 0$ and
    \item $\sum_{k = 0}^n (k + 1) \binom{n}{k} = n 2^{n - 1}$.
  \end{enumerate}
\end{corollary}
\begin{proof}
  \begin{enumerate}
    \item Let $x = -1$. We may notice that, by \Cref{theorem:binomial}, 
      \[
        \sum_{k = 0}^n (-1)^k \binom{n}{k} = 
        \sum_{k = 0}^n 1^{n - k} x^k \binom{n}{k} = (1 - 1)^n = 0.
      \]
    \item This equality is a bit more tricky. Let $x = 1$. Note that
      \begin{multline*}
        \sum_{k = 0}^n (k + 1) \binom{n}{k} = 
        \sum_{k = 0}^n (k + 1) x^k 1^{n - k} \binom{n}{k} = \\
        \frac{d \sum_{k = 0}^n x^k \binom{n}{k}}{dx} = n (1 + x)^{n - 1}.
      \end{multline*}
      As a result, $\sum_{k = 0}^n (k + 1) \binom{n}{k} = n 2^n$.
  \end{enumerate}
\end{proof}


Using the idea of the second equality, we can give an alternative --- more
explicit --- proof of \Cref{claim:guess-one-out-of-many}.
\begin{proof}[Proof of \Cref{claim:guess-one-out-of-many}]
\label{proof-claim:guess-one-out-of-many}
  Let $x = 1 / 2$. Then the value we would like to compute is 
  $\sum_{k = 1}^n k x^k$. Note that
  \begin{multline*}
    \sum_{k = 1}^n k x^k = x \sum_{k = 1}^n k x^{k - 1} =
    x \frac{d \sum_{k = 0}^n x^k}{dx} = \\
    x \frac{d}{dx} \frac{1 - x^{n + 1}}{1 - x} = 
    x \frac{-(n + 1) x^n (1 - x) + (1 - x^{n + 1})}{(1 - x)^2} =  \\
    2\left(
      1 - \frac{1}{2^{n + 1}} - (n + 1) \frac{1}{2^{n + 1}}
    \right) = 2 - \frac{n + 2}{2^n}.
  \end{multline*}
\end{proof}

\subsection{Counting Groups of Subsets}
In this section we study a generalization of the question we study in the
previous sections: ``How many ways to select $\ell$ groups made of $k_1$, $k_2$,
\dots, $k_\ell$ objects, respectively, out of $n$''. We denote this number by
$\binom{n}{k_1 \ k_2 \ \dots \ k_\ell \  (n - m)}$, where
$m = k_1 + \dots + k_\ell$.

Clearly selecting these objects is the same as selecting $k_1$ objects out of
$n$, after that selecting $k_2$ objects out of $n - k_1$ etc. As a result,
\begin{multline*}
  \binom{n}{k_1 \ k_2 \ \dots \ k_\ell \ (n - m)} = \\
  \frac{n!}{k_1! (n - k_1)!} \cdot \frac{(n - k_1)!}{k_2! (n - k_1 - k_2)!}
  \cdot \ldots \cdot
  \frac{
    (n - k_1 - k_2 - \dots - k_{\ell - 1})!
  }{
    k_\ell! (n - k_1 - k_2 - \dots - k_\ell)!
  } = \\
  \frac{n!}{k_1! k_2! \dots k_\ell! (n - k_1 - k_2 - \dots - k_\ell)!}.
\end{multline*}

Similarly to the Binomial theorem, we can prove the following.
\begin{theorem}[Multinomial theorem]
\label{theorem:multinomial-coefficients}
  For any real numbers $x_1$, $x_2$, \dots, $x_\ell$ and integer $n$,
  \[
    (x_1 + x_2 + \dots + x_\ell)^n =
    \sum_{k_1, k_2, \dots, k_\ell ~:~ k_1 + k_2 + \dots + k_\ell = n}
      \binom{n}{k_1 \ k_2 \ \dots \ k_\ell} \prod_{i = 1}^n x_i^{k_i}.
  \]
\end{theorem}

\begin{exercise}
  Prove Theorem~\ref{theorem:multinomial-coefficients}.
\end{exercise}

\section{Double Counting}
\marginurl{%
  Double Counting:\\\noindent
  Introduction to Combinatorics \#4
}{youtu.be/OrzMP8nuuho}
The method that was used to prove Theorem~\ref{theorem:pascals-rule} can be
generalized to a method that is called \emph{double counting principle}.
The double counting principle states the following “obvious” fact: if the
size of a set is counted in two different ways, the answers are the same.

Using this principle we may prove the following theorem.
\begin{theorem}[Vandermonde's identity]
  For any integers $n, m > k$,
  $\sum_{i = 0}^k \binom{n}{i} \binom{m}{k - i} = \binom{n + m}{k}$.
\end{theorem}
\begin{proof}
  The idea is as follows, let us imagine that we have $n$ parrots and $m$ crows,
  and we need to find how many ways to select $k$ birds.
  It is easy to see that it is equal to $\binom{n + m}{k}$. At the same
  if we need to select $i$ parrots there are $\binom{n}{i}\binom{m}{k - i}$
  ways to do this. Thus the number is also equal to
  $\sum_{i = 0}^k \binom{n}{i}\binom{m}{k - i}$.
\end{proof}

However, the method can be used in a more sophisticated way.
\begin{lemma}[Handshaking Lemma]
\label{lemma:handshaking}
  Suppose some number of people meet at a party and some shake hands. Assume
  that no person shakes his or her own hand and furthermore no two people shake
  hands more than once.

  The number of guests who shake hands an odd number of times is even.
\end{lemma}
\begin{proof}
  Let $1$, \dots, $n$ be the people at the party. We apply double counting to
  the set of ordered pairs $(i, j)$ for which $i$ and $j$ shake hands with each
  other at the party. Let $d_i$ be the number of times that $i$ shakes hands,
  and $e$ be the total number of handshakes that occur. On one hand, the number
  of pairs is $\sum_{i = 1}^n d_i$, since for each $i$ the number of choices of
  $j$ is equal to $d_i$. On the other hand, each handshake gives rise to two
  pairs $(i, j)$ and $(j, i)$; so the total is $2e$.
  Thus $\sum_{i = 1}^n d_i = 2e$. But, if the sum of $n$ numbers is even, then
  evenly many of the numbers are odd.
  (Because if we add an odd number of odd numbers and any number of even
  numbers, the sum will be always odd).
\end{proof}


\begin{chapterendexercises}
  \exercise Let $m$ be some integer. Show that product of $m$ consecutive
    integers is divisible by $m!$.
    \begin{solution}
     In other words we need to show that 
     $\frac{n \cdot (n + 1) \cdot \dots \cdot (n + m - 1)}{m!}$ is an integer,
     for any integer $n$. But one may notice that 
     $\frac{n \cdot (n + 1) \cdot \dots \cdot (n + m - 1)}{m!} = 
      \binom{n + m - 1}{m}$ which is an integer.
    \end{solution}
  \exercise Show that
    $\numberOfPermutations[n]{x + y} = \sum_{k = 0}^n \binom{n}{k}
      \numberOfPermutations[k]{x} \numberOfPermutations[n - k]{y}$.
  \exercise Show that
    $\sum_{k = 0}^n \binom{n}{k}^2 = \binom{2n}{n}$.
  \exercise Show that $\sum_{m = k}^n \binom{m}{k} =
    \binom{n + 1}{k + 1}$.
    \hint{Note that the formula on the right corresponds to the number of ways
      to select $k + 1$ elements out of $n + 1$; $m$ in the summation on the
      left denotes the maximum of this selected set minus one.}
  \exercise Using the previous formula, find the formulas for the following
    expressions:
    \begin{enumerate*}
      \item $\sum_{k = 0}^n k$,
      \item $\sum_{k = 0}^n k^2$, and
      \item $\sum_{k = 0}^n k^3$.
    \end{enumerate*}

  \exercise Show that 
    $\sum_{k = 2}^n k (k - 1) \binom{n}{k} = n (n - 1) 2^{n - 2}$ for any
    positive integer $n > 2$.
    \begin{solution}
      There are several possible solutions. We present two of them: the first
      one is similar to \Cref{corollary:binommials-equality}, and the second is
      based on the combinatorual meaning of the equality.
      \begin{enumerate}
        \item Note that $k (k - 1) \binom{n}{k} = 
          k (k - 1) \frac{n!}{k! (n - k)!} = n (n - 1) \binom{n - 2}{k - 2}$.
          Thus
          \begin{multline*}
            \sum_{k = 2}^n k (k - 1) \binom{n}{k} =
            \sum_{k = 2}^n n (n - 1) \binom{n - 2}{k - 2} =  \\
              n (n - 1) \sum_{k = 0}^{n - 2} \binom{n - 2}{k} = 
              n (n - 1) 2^{n - 2}.
          \end{multline*}
        \item Another solution is the following. Imagine that we have $n$ people in a
          group and we need to choose a subgroup of them, a head of this subgroup, and
          a vice-head of this group.

          It is easy to see that we have $n$ ways to select the head, $n - 1$
          ways to select the vice-head, and $2^{n - 2}$ ways
          to select the rest of the subgroup. On the other hand, if we know that the
          subgroup has $i$ members, then there are $\binom{n}{i}$ ways to select the
          subgroup, $i$ ways to select the head, and $i - 1$ ways to select the
          vice-head, thus there are $\sum_{k = 2}^n k (k - 1) \binom{n}{k}$ ways to
          select the subgroup, its leader, and its vice-leader. As a result,
          $n (n - 1) 2^{n - 2} = \sum_{k = 2}^n k (k - 1) \binom{n}{k}$.
      \end{enumerate}
    \end{solution}


  \exercise Using the binomial theorem, explain the following equalities:
    \begin{enumerate*}
      \item $\sum_{k = 0}^{n} \binom{2n}{2k} =
        \sum_{k = 0}^{n - 1} \binom{2n}{2k + 1}$,
        and
      \item $\sum_{k = 0}^{n} \binom{2n + 1}{2k} =
        \sum_{k = 0}^n \binom{2n + 1}{2k + 1}$.
    \end{enumerate*}

  \exercise[recommended] Show that $\sum_{k = 0}^n \binom{m + k}{k} =
    \binom{m + n + 1}{n}$.
  \exercise Show that $\sum_{k = 0}^n \binom{n - k}{k} = f_{n + 1}$,
    where $f_1 = 1$, $f_2 = 1$, and $f_{n + 2} = f_{n + 1} + f_n$ for $n > 0$.
  \exercise Show that $\binom{n}{m} \binom{m}{k} =
    \binom{n}{k} \binom{n - k}{m - k}$.
  \exercise[recommended] Show that
    $(a + 1)^p \equiv a^p + 1 \pmod{p}$.
    \hint{Use the binomial theorem.}
  \exercise[recommended] We say that a function 
    $f : \set{0, 1}^n \to \set{0, 1}$ depends on the $i$th argument iff for some
    $a_1, \dots, a_{i - 1}, a_{i + 1}, \dots, a_n \in \set{0, 1}$
    \[
      f(a_1, \dots, a_{i - 1}, 0, a_{i + 1}, \dots, a_n) \neq
      f(a_1, \dots, a_{i - 1}, 1, a_{i + 1}, \dots, a_n).
    \]
    We also say that the function $f$ depends on all the arguments iff for all
    $i \in \range{n}$ it depends on $i$th argument.

    Find the number of functions $f : \set{0, 1}^n \to \set{0, 1}$ depending on all
    arguments.

    \begin{solution}
      Let $S \subseteq \range{n}$ and $F_i$ be the set of function from $\set{0,
      1}^n$ to $\set{0, 1}$ not depending on the $i$th input. It is easy to see
      that there are $2^{2^{n - 1}}$ elements in $F_i$ and moreover, there are
      $2^{2^{n - \cardinality{S}}}$ elements in the set $\cap_{i \in S} F_i$.
      Thus by the inclusion-exclusion principle, there are
      \[
        \sum_{S \subseteq [n] ~:~ S \neq \emptyset} 
          (-1)^{\cardinality{S} + 1} 2^{2^{n - \cardinality{S}}} =
        \sum_{k = 1}^n (-1)^k 2^{2^{n - k}} \binom{n}{k}
      \]
      functions not depending on at least one argument.
      As a result, the answer is $2^{2^n} -
      \sum_{k = 1}^n (-1)^k 2^{2^{n - k}} \binom{n}{k}$.
    \end{solution}
  \exercise Find the largest coefficient of $(x_1 + x_2 + \dots + x_k)^k$.
  \exercise Prove that,
    without using Theorem~\ref{theorem:multinomial-coefficients},
    \[
      \sum_{k_1, k_2, k_3 ~:~ k_1 + k_2 + k_3 = n} \binom{n}{k_1 \ k_2 \ k_3} =
      3^n.
    \]
\end{chapterendexercises}
