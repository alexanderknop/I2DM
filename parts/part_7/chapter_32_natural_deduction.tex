\chapter{Natural Deduction For Propositional Logic}
\label{chapter:propositional-deduction}
\marginurl{%
  Natural Deduction:\\\noindent
  Introduction to Mathematical Logic \#3
}{youtu.be/PfVafyptFtM}

There are two issues with thr method discussed in
\Cref{chapter:propositional-truth}: first to check a semantic implication 
we need to consider \textbf{all} possible values of the variables, secondly this
method does not look like the proofs we use in mathematics.

In this chapter we introduce the way to check semantic implications that looks 
more similar to real-life proofs.
Let us illustrate this using the following example. Imagine that we know that
$\lnot q$, $p \limplies q$. Using the contraposition argument and modus ponens
we may derive $\lnot p$. Indeed, by contraposition we may conclude that 
$\lnot q \limplies \lnot p$ and modus ponens implies that $\lnot p$ is true
since $\lnot q$ is true.

In other words, we can combine several tautologies to prove another tautology.
Apparently it is enough to fix some small number of tautologies to derive all
other tautologies, we call these tautologies ``rules''. There are several ways
to write such proofs, we are going to use Fitch notation for natural deduction.
In this notation any proof is written in several rows, each row in a Fitch-style
proof is either:
\begin{itemize}
  \item an assumption or subproof assumption.
  \item a sentence justified by the citation of
    \begin{enumerate*}[label=(\roman*)]
      \item a rule of inference and
      \item the prior line or lines of the proof that license that rule.
    \end{enumerate*}
\end{itemize}
We say that there is a natural deduction derivation of $\phi$ from $\psi_1$,
\dots, $\psi_k$. If there is a Fitch-style proof starting with the assumptions
$\psi_1$, \dots, $\psi_k$, and finishes with the formula $\phi$.
Using this scheme we may write the argument we just mentioned as follows.

\noindent $
  \begin{nd}
    \hypo {1} {\lnot q}
    \hypo {2} {p \limplies q}
    \have {3} {\lnot q \limplies \lnot p} \by{contraposition}{2}
    \have {4} {\lnot p} \by{modus ponens}{1, 3}
  \end{nd}
$

\noindent In the rest of the section we are going to list all the rules we use.

\paragraph{Conjunctions.}
In order to introduce a conjunction we can use the following rule.
\[
  \begin{nd}
    \have [m] {1} {A}
    \have [n] {3} {B}
    \have [~] {5} {A \land B} \ai{1, 3}
  \end{nd}
\]
This rule corresponds to the tautology $(A \land B) \limplies (A \land B)$.

In order to eliminate conjunctions we can use the following two rules.
\begin{center}
  \begin{tabular}{c c}
    $\begin{nd}
      \have [m] {1} {A \land B}
      \have [~] {3} {A} \ae{1}
    \end{nd}$
    &
    $\begin{nd}
      \have [m] {1} {A \land B}
      \have [~] {3} {B} \ae{1}
    \end{nd}$
  \end{tabular}
\end{center}
These rules correspond to the tautologies $(A \land B) \limplies A$ and
$(A \land B) \limplies B$.

\paragraph{Disjuctions.}
In order to introduce a disjunction we can use the following two rules.
\begin{center}
  \begin{tabular}{c c}
    $\begin{nd}
      \have [m] {1} {A}
      \have [~] {3} {A \lor B} \oi{1}
    \end{nd}$
    &
    $\begin{nd}
      \have [m] {1} {A}
      \have [~] {3} {B \lor A} \oi{1}
    \end{nd}$
  \end{tabular}
\end{center}
These rules correspond to the tautologies $A \limplies (A \lor B)$ and
$A \limplies (B \lor A)$.

In order to eliminate a disjunction we can use the following rule.
\[
  \begin{nd}
    \have [m] {1} {A \lor B}
    \open
      \hypo [i] {3} {A}
      \have[j] {5} {C}
    \close
    \open
      \hypo [k] {6} {B}
      \have[l] {8} {C}
    \close
    \have[~] {9} {C} \oe{1, 3-5, 6-8}
  \end{nd}
\]
This rule corresponds to the tautology
$\bigl( (A \lor B) \land (A \limplies C) \land (B \limplies C) \bigr)
\limplies C$.

\paragraph{Implications.}
In order to introduce an implication we can use the following two rules.
\[
  \begin{nd}
    \open
      \hypo [i] {3} {A}
      \have[j] {5} {B}
    \close
    \have[~] {9} {A \limplies B} \ii{3-5}
  \end{nd}
\]
This rule corresponds to the tautology
$(A \limplies B) \limplies (A \limplies B)$.

In order to eliminate an implication we can use the following rule.
\[
  \begin{nd}
    \have [m] {1} {A \limplies B}
    \have [n] {2} {A}
    \have[~] {9} {B} \ie{1, 2}
  \end{nd}
\]
This rule corresponds to the tautology
$\bigl( (A \limplies B) \land A \bigr)
\limplies B$.

\paragraph{Negations.}
In order to introduce a negation we can use the following two rules ($\perp$ is
a special symbol representing a false statement).
\[
  \begin{nd}
    \open
      \hypo [i] {3} {A}
      \have[j] {5} {\perp}
    \close
    \have[~] {9} {\lnot A} \ni{3-5}
  \end{nd}
\]
This rule corresponds to the tautology
$\bigl( A \limplies \perp \bigr)
\limplies \lnot A$.

In order to eliminate a negation we can use the following rule.
\[
  \begin{nd}
    \have [m] {1} {A}
    \have [n] {2} {\lnot A}
    \have[~] {9} {\perp} \ne{1, 2}
  \end{nd}
\]
This rule corresponds to the tautology
$\bigl( A \land \lnot A \bigr)
\limplies \perp$.

\paragraph{Truths and falsities.}
Additionally, we have the following two rules.
\begin{center}
  \begin{tabular}{c c}
    $\begin{nd}
      \have [m] {1} {\perp}
      \have [~] {3} {A} \be{1}
    \end{nd}$
    &
    $\begin{nd}
      \open
        \hypo [i] {3} {\lnot A}
        \have[j] {5} {\perp}
      \close
      \have[~] {9} {A} \by{IP}{3, 5}
    \end{nd}$
  \end{tabular}
\end{center}

\marginurl{%
  An online tool to check natural deduction proofs
}{proofs.openlogicproject.org/}

\begin{exercise}
  Check that all the tautologies we mentioned are indeed tautologies.
\end{exercise}

\section{Examples of Derivations}
In this section we give several derivations using the rules we just introduced.

First, we prove that if we know that $A \limplies \lnot A$ we can derive that
$\lnot A$.

\noindent $
  \begin{nd}
    \hypo {1} {A \limplies \lnot A}
    \open
      \hypo {2} {A}
      \have {3} {\lnot A} \ie{1, 2}
      \have {4} {\perp} \ne{2, 3}
    \close
    \have {5} {\lnot A} \ni{2-4}
  \end{nd}
$

Another statement we are going to prove is that if
$A \limplies (A \land \lnot A)$ is true, then $\lnot A$ is also true.

\noindent $
  \begin{nd}
    \hypo {1} {A \limplies (A \land \lnot A)}
    \open
      \hypo {2} {A}
      \have {3} {A \land \lnot A} \ie{1, 2}
      \have {4} {\lnot A} \ae{3}
      \have {5} {\perp} \ne{2, 4}
    \close
    \have {6} {\lnot A} \ni{2-5}
  \end{nd}
$

A bit more complicated is the proof of the law of excluded middle:
$A \lor \lnot A$.

\noindent $
  \begin{nd}
    \hypo {1} {}
    \open
      \hypo {2} {\lnot (A \lor \lnot A)}
      \open
        \hypo {3} {A}
        \have {4} {A \lor \lnot A} \oi{3}
        \have {5} {\perp} \ne{2, 4}
      \close
      \have {6} {\lnot A} \ni{3-5}
      \have {7} {A \lor \lnot A} \oi{6}
      \have {8} {\perp} \ne{2, 8}
    \close
    \have {9} {A \lor \lnot A} \by{IP}{2-8}
  \end{nd}
$

\section{Soundness and Completeness}
\marginurl{%
  Soundness and Completeness:\\\noindent
  Introduction to Mathematical Logic \#4
}{youtu.be/9Utsppn-M_I}
The most important properties of the natural deduction are the following two
theorems.

\begin{theorem}[completeness of natural deductions]
  Let $\phi$ be a propositional formula. If $\phi$ is a tautology, then
  there is a proof of $\phi$. Moreover if $\Sigma$ is a finite set of
  propositional formulas and $\Sigma \models \phi$, then there is a
  derivation of $\phi$ from $\Sigma$.
\end{theorem}

\begin{theorem}[soundness of natural deductions]
  Let $\phi$ be a propositional formula. If there is a proof of $\phi$, then
  $\phi$ is a tautology. Moreover if $\Sigma$ is a finite set of
  propositional formulas and there is a derivation of $\phi$ from $\Sigma$,
  then $\Sigma \models \phi$.
\end{theorem}


Proofs of these two theorems are not that difficult but very technical. So
we will leave these theorems without proofs and just give some ideas behid 
their proofs.

\paragraph{Completeness of natural deductions.}
The proof of this statement exploits the following idea: if a propositional formula
is a tautology, then we can verify this statement using the truth table. So
the proof may simply brute-force all the values of the variables of a formula
and check that the formula is indeed true.

For example, consider a tautology $(\lnot A \land \lnot B) \limplies \lnot (A \lor B)$.
The proof of this tautology is as follows.
First we derive $A \lor \lnot A$ and $B \lor \lnot B$, and we use these two
formulas to consider cases using the elimination of disjunction.

\noindent$
\begin{nd}
  \hypo {1} {}
  \have {2} {A \lor \lnot A} \by{the law of excluded middle}{}
  \have {3} {B \lor \lnot B} \by{the law of excluded middle}{}
\end{nd}
$


\noindent After that, we consider the case when $A$ and $B$ are both true. Note
that the assumption of the implication is false in this case. Thus, we just need
to assume $\lnot A \land \lnot B$, derive the contradiction, and derive
$\lnot (A \lor B)$.

\noindent$
\begin{ndresume}
  \open
    \hypo {4} {A}
    \open
      \hypo {5} {B}
      \open
        \hypo {6} {\lnot A \land \lnot B}
        \have {7} {\lnot A} \ae{6}
        \have {8} {\perp} \ne{4, 7}
        \have {9} {\lnot (A \lor B)} \be{8}
      \close
      \have {10} {(\lnot A \land \lnot B) \limplies \lnot (A \lor B)} \ii{6-9}
    \close
\end{ndresume}
$


\noindent After that, we consider the case when $A$ is true but and $B$ is
false. In this case, the assumption of the implication is also false; thus, the
proof is the same as in the previous case.

\noindent$
\begin{ndresume}
    \open
      \hypo {11} {\lnot B}
      \open
        \hypo {12} {\lnot A \land \lnot B}
        \have {13} {\lnot A} \ae{12}
        \have {14} {\perp} \ne{4, 13}
        \have {15} {\lnot (A \lor B)} \be{14}
      \close
      \have {16} {(\lnot A \land \lnot B) \limplies \lnot (A \lor B)} \ii{6-9}
    \close
    \have {17} {(\lnot A \land \lnot B) \limplies \lnot (A \lor B)} \oe{2, 5-10, 11-16}
  \close
\end{ndresume}
$

\noindent The third case is when $A$ is false and $B$ is
true. In this case the assumption of the implication is false again, thus the
proof is the same as in the previous two cases.

\noindent$
\begin{ndresume}
  \open
    \hypo {18} {\lnot A}
    \open
      \hypo {19} {B}
      \open
        \hypo {20} {\lnot A \land \lnot B}
        \have {21} {\lnot B} \ae{20}
        \have {22} {\perp} \ne{19, 22}
        \have {23} {\lnot (A \lor B)} \be{22}
      \close
      \have {24} {(\lnot A \land \lnot B) \limplies \lnot (A \lor B)} \ii{20-23}
    \close
\end{ndresume}
$

\noindent Finally, we consider the case when $A$ and $B$ are false. In this
case the assumption of the implication is true, and since the formula is a
tautology and $\lnot A \land \lnot B$ is true, we know that $\lnot (A \lor B)$
is also true. Assume that $A \lor B$ is true and note that this is impossible.
Thus using introduction of the negation we can prove the statement.

\noindent$
\begin{ndresume}
    \open
      \hypo {25} {\lnot B}
      \open
        \hypo {26} {\lnot A \land \lnot B}
        \open
          \hypo {27} {A \lor B}
          \open
            \hypo {28} {A}
            \have {29} {\perp} \ne{18, 28}
          \close
          \open
            \hypo {30} {B}
            \have {31} {\perp} \ne{25, 30}
          \close
          \have {32} {\perp} \oe{27, 28-29, 30-31}
        \close
        \have {33} {\lnot (A \lor B)} \ne{26-32}
      \close
      \have {39} {(\lnot A \land \lnot B) \limplies \lnot (A \lor B)} \ii{26-33}
    \close
    \have {40} {(\lnot A \land \lnot B) \limplies \lnot (A \lor B)} \oe{1, 3-17, 18-39}
  \close
\end{ndresume}
$

\paragraph{Soundness of natural deductions.}
The idea behind the proof of the soundness is also simple. We just explain that
every line of the proof represent a tautology, including the last one. We
illustrate this on the exaple of the proof of $A \lor \lnot A$. Recall that the
proof of this tautology is the following.

\noindent $
  \begin{nd}
    \hypo {1} {}
    \open
      \hypo {2} {\lnot (A \lor \lnot A)}
      \open
        \hypo {3} {A}
        \have {4} {A \lor \lnot A}
        \have {5} {\perp}
      \close
      \have {6} {\lnot A}
      \have {7} {A \lor \lnot A}
      \have {8} {\perp}
    \close
    \have {9} {A \lor \lnot A}
  \end{nd}
$

\begin{enumerate}
  \item The second line is just an assumption, so the corresponding tautology is
    $\lnot (A \lor \lnot A) \limplies \lnot (A \lor \lnot A)$.
  \item Line~3 is also an assumption so the corresponding tautology is
    $\lnot (A \lor \lnot A) \limplies (A \limplies A)$.
  \item Line~4 is a formula $A \lor \lnot A$ which we derived under assumptions
    $\lnot (A \lor \lnot A)$ and $A$, so the corresponding tautology is
    $\lnot (A \lor \lnot A) \limplies (A \limplies (A \lor \lnot A))$ (it is a
    tautology since we replaced $A$ by $A \lor \lnot A$ in the conclusion of
    the formula corresponding to Line~3).
  \item Line~5 is the formula $\perp$ which we derived under assumptions
    $\lnot (A \lor \lnot A)$ and $A$, so the corresponding tautology is
    $\lnot (A \lor \lnot A) \limplies (A \limplies \perp)$ (it is a
    tautology since on Line~4 we explained that $\lnot (A \lor \lnot A)
    \limplies (A \limplies (A \lor \lnot A))$).
  \item Line~6 is a formula $\lnot A$ which we derived under assumptions
    $\lnot (A \lor \lnot A)$, so the corresponding tautology is
    $\lnot (A \lor \lnot A) \limplies \lnot A$ (it is a
    tautology since on Line~5 we explained that $A \limplies \perp$ under the
    assumption $\lnot (A \lor \lnot A)$).
  \item Line~7 is a formula $A \lor \lnot A$ which we derived under assumptions
    $\lnot (A \lor \lnot A)$, so the corresponding tautology is
    $\lnot (A \lor \lnot A) \limplies (A \lor \lnot A)$ (it is a
    tautology since on Line~6 we explained that $A$ under the
    assumption $\lnot (A \lor \lnot A)$).
  \item Line~8 is a formula $\perp$ which we derived under assumptions
    $\lnot (A \lor \lnot A)$, so the corresponding tautology is
    $\lnot (A \lor \lnot A) \limplies \perp$ (it is a
    tautology since on Line~6 we explained that $A \lor \lnot A$ under the
    assumption $\lnot (A \lor \lnot A)$).
  \item Finally, Line~9 is a formula $A \lor \lnot A$ (it is a tautology since
    we proved that $\lnot (A \lor \lnot A) \limplies \perp$ is a tautology)
\end{enumerate}



\begin{chapterendexercises}
  \exercise  Write a natural deduction derivation of $A \lor C$ from
    hypothesis $(A \land B) \lor C$.
  \exercise Write a natural deduction derivation of $B \lor C$ from
    hypothesis $A \limplies B$ and $\lnot A \limplies C$.
  \exercise Write a natural deduction derivation of
    $(W \lor Y) \limplies (X \lor Z)$ from
    hypotheses $W \limplies X$ and $Y \limplies Z$.
  \exercise Let us formulate the pigeonhole principle using propositional
    formulas. Let
    $V = \set{x_{1, 1}, \dots, x_{n + 1, 1}, x_{1, 2} \dots, x_{n + 1, n}}$
    (informally $x_{i, j}$ is true iff the $i$th pigeon is in the $j$th hole).
    Consider the following propositional formulas on the variables from
    $V$.
    \begin{itemize}
      \item $L_i$ ($i \in [n + 1]$) is equal to $\bigvee_{j = 1}^n x_{i, j}$.
        (Informally this formula says that the $i$th pigeon is in a hole.)
      \item $R_j$ ($j \in [n]$) is equal to
        $\biglor_{i_1 = 1}^{n + 1} \biglor_{i_2 = i_1 + 1}^{n + 1}
        (x_{i_1, j} \land x_{i_2, j})$.
        (Informally this formula says that there are two pigeons in the $j$th
        hole.)
      \end{itemize}

      Show that there is a natural deduction derivation of
      $\left(
          \bigland_{i = 1}^{n + 1} L_i
        \right)
        \limplies
        \left(
          \biglor_{i = 1}^{n} R_i
        \right)$.
    \exercise In this exercise we think about clauses as sets of literals so
      the order of disjunctions and repetitions of literals are not important.
      We say that a clause $C$ can be obtained from clauses $A$ and $B$
      using the \emph{resolution} rule if $C = A^\prime \lor B^\prime$,
      $A = x \lor A^\prime$, and $B = \lnot x \lor B^\prime$, for some variable
      $x$.

      We say that a clause $C$ can be derived from clauses $A_1$, \dots, $A_m$
      using resolutions
      if there is a sequence of clauses $D_1$, \dots, $D_\ell = C$ such that
      each $D_i$
      \begin{itemize}
        \item is either obtained from clauses $D_j$ and $D_k$ for $j, k < i$ using the
          \emph{resolution} rule, or
        \item is equal to $A_j$ for some $j \in [m]$, or
        \item is equal to $D_j \lor E$ for some $j < i$ and a clause $E$.
      \end{itemize}

      Show that if an empty clause $\perp$ can be derived from clauses $A_1$,
      \dots, $A_m$ using the resolution rule, then $A_1$, \dots, $A_m$
      semantically imply $\perp$.
\end{chapterendexercises}
