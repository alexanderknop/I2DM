\chapter{P and N Positions}
In this part we use our knowledge abot basics of mathematical reasoning
to study games similar to chekers, chess, shogi, and tic tac toe. The games we
are going to study are called combinatorial games. In these games there are two
players, each know all the information, there are no chance moves, and whenthe game ends there is always a winner\footnote{%
  The last condition implies that among the beforementioned games only chekers
  are combinatorial since all of them allow draws; however, we may change the
  rules to disallow the draws and this change would make all of them
  combinatorial.
}. Such a game is determined by a set of positions, and possible moves
from each position for each player. Usually, players are taking turns until
they reach a position such that no moves are possible and one of the player is
declared a winner.

Since chess, shogi and even tic tac toe are realtively complicated,
we are going to start from much simpler example of combinatorial games.
\begin{game}
  In this game there are two players I and II.\
  \begin{itemize}
    \item They have a pile of 21 chips.
    \item They make moves in turns with player I starting,
      each move consists of moving one, two or three chips out of the pile.
    \item The player that removes the last chip wins.
  \end{itemize}
\end{game}
The quesiton we would like to answer is there a strategy for one of the players
to always win? So in the rest of this part we assume that both players are
playing optimally; i.e., if there is a winning strtegy they follow the strategy.

To analyze this game we need the following two observations:
\begin{enumerate}
  \item the game is symmetric and the only difference between the players is
    who makes the first move, and
  \item if at some point the players have $n$ chips it does not matter how they
    achieved this, it will not affect the rest of the game.
\end{enumerate}
Using this remarks and induction (this style of induction is sometimes
refered as \emph{backward induction}) we are abel to analise the game.

Let us consider some certain states of the game.
Assume that they have at most $3$ chips left, in this case the player that make
the move wins. However, if there are $4$ chips, the
player that makes the first move should always take at least $1$ chip so it
loses since after its turn there are at most $3$ chip. Similarly, if there
are $5$ chips, the first player to move wins since it can take a chip and
make the second player to start with $4$.

So twe can formulate the following conjecture.
Assume that $n$ chips left in the pile. Let $r$ be the reminder of $n$ modulo
$4$. Then if $r = 0$, the first player to move loses, otherwise, the other
player loses.

Let us prove this using induction. We already proved the base case so
we need to prove the induction step from $n$ to $n + 1$.
\begin{itemize}
  \item If $n \equiv 0 \pmod{4}$, then the first player to move can remove one
    chip and the other player will start with $n$ chips so by the induction
    hypothesis he/she loses.
  \item If $n \equiv 1 \pmod$. then the first player to move can remove two
    chips and the other player will start with $n$ chips so by the induction
    hypothesis he/she loses.
  \item If $n \equiv 2 \pmod$. then the first player to move can remove three
    chips and the other player will start with $n$ chips so by the induction
    hypothesis he/she loses.
  \item If $n \equiv 3 \pmod$. then after the current player move the other
    player will start with either $n$, or $n - 1$, or $n - 2$ chips. But all
    these numbers have non-zero reminders modulo $4$. So the other player
    can win in any case.
\end{itemize}

To study combinatorial games we need to give a formul definition of them.
\begin{definition}
  A game is combinatorial if
  \begin{itemize}
    \item there are two players,
    \item there is a set of possible positions in the game,
    \item for each position and each player, there is a fixed set of possible
      legal moves,
    \item players alternate moving,
    \item the game ends when no moves are possible for the player whose
      turn is to move.
  \end{itemize}
  There are possible winning conditions,
  \begin{description}
    \item [normal play rule:] the player that made the last move wins, and
    \item [misere play rule:] the player that made the last move loses.
  \end{description}
  If the game never ends, we declare a draw. If the game always ends, we
  say that the game satisfies \emph{the ending condition}.

  If the possible moves are the same for both players the game is
  called \emph{impartial} otherwise it is called \emph{partizan}.
\end{definition}

Note that these games do not allow random moves, hidden information,
simultaneous moves, and a draw in a finite number of steps so
pocker, battleships, rock-paper-scissors, and tick tack toe are not
combinatorial games.
