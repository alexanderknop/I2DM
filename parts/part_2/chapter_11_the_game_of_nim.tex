\chapter{The Game of Nim}
\marginurl{%
    The game of Nim:\\\noindent
    Introduction to Combinatorial Game Theory \#3
}{youtu.be/H-SyB0NK3H8}

This chapter discusses probably the most famous combinatorial game, the game
of \emph{Nim}.
In this game there are several piles of chips on the table. On each turn
the current player may remove some number of chips from \emph{one} of the piles;
however, the player should remove \emph{at least one chip}.
We say that a game of Nim is a $k$-pile game of Nim if there are $k$ piles.
\marginurl{%
  You can play Nim on this website
}{dotsphinx.com/games/nim/}

We start from analysis of the game when we have one pile of chips. It is clear
that the first player to move wins since he/she may remove all the chips.

Consider a more complicated case when we have two piles of size $n$ and $m$
respectively. We need to consider two cases:
\begin{enumerate}
  \item If $n = m$, then the second player to move wins. Indeed, we can
    use the symmetric strategy; i.e., if the first player removes $s$
    chips from one pile we also remove $s$ chips from the other pile.
    It is clear that we can always make a move as long as the first player can.
  \item Otherwise, the first player wins because it can move to the state
    with two equal piles.
\end{enumerate}

The case of three piles is even more complicated. So we spend the rest of the
chapter studying it.

\section{Nim Sum}

We start from a definition of the XOR operation
$\lxor : \set{0, 1} \times \set{0, 1} \to \set{0, 1}$,
also known as ``exclusive or''), this operation is defined as follows:
$a \lxor b = 1$ iff $a \neq b$.

It is well-known that any number $n \in \N_0$
can be represented as a binary number ($\N_0$ denotes nonnegative integers);
\nomenclature[S]{$\N_0$}{denotes the set of nonnegative integers}
we write $n = (a_\ell, \dots, a_0)_2$ if $n = \sum_{i = 0}^\ell a_i 2^i$.
For example,
$5 = 4 + 1 = 1 \cdot 2^2 + 0 \cdot 2^1 + 1 \cdot 2^0 = (1, 0, 1)_2$ and
$6 = 4 + 2 = 1 \cdot 2^2 + 1 \cdot 2^1 + 0 \cdot 2^0 = (1, 1, 0)_2$.
So we can define the Nim sum $\bitwisexor : \N_0 \times \N_0 \to \N_0$, also
known as bitwise xor, as follows:
$(a_\ell, \dots, a_0)_2 \bitwisexor (b_\ell, \dots, b_0)_2 =
    (a_\ell \lxor b_\ell, \dots, a_0 \lxor b_0)_2$.
For example, $5 \bitwisexor 6 = (1, 0, 1)_2 \bitwisexor (1, 1, 0)_2 =
(1 \lxor 1, 0 \lxor 1, 1 \lxor 0)_2 = (0, 1, 1)_2$.
\nomenclature[N]{$\bitwisexor$}{denotes the Nim sum}

\begin{exercise}
  Show that $a \bitwisexor (b \bitwisexor c) = (a \bitwisexor b) \bitwisexor c$
  for any $a, b, c \in \N_0$.
\end{exercise}
Hence, we are going to write $a \bitwisexor b \bitwisexor c$ instead of
$a \bitwisexor (b \bitwisexor c)$ and $(a \bitwisexor b) \bitwisexor c$.

\section{Bouton's Theorem}

Now we may notice that $a \bitwisexor b = 0$ iff $a = b$. So
our result about $2$-pile Nim can be rephrased:
a position $(a, b)$ in the $2$-pile Nim is a P-position iff
$a \bitwisexor b = 0$. Which leads us to the next theorem.
\begin{theorem}[Bouton]
\label{theorem:bouton}
  A position $(a, b, c)$ in $3$-pile Nim is a P-position iff
  $a \bitwisexor b \bitwisexor c = 0$
\end{theorem}
\begin{proof}
  We prove the statement using structural induction.
  First note that the only terminal position the $3$-pile Nim is $(0, 0, 0)$
  and $(0, 0, 0)$ and $0 \bitwisexor 0 \bitwisexor 0 = 0$.

  Let us consider
  some $(a, b, c)$ such that $a \bitwisexor b \bitwisexor c \neq 0$.
  We need to show that there is a move from this position to a P-position.
  Let $a \bitwisexor b \bitwisexor c =
      (0, \dots, 0, 1, r_{k - 1}, \dots, r_0)_2$. So among $a$, $b$, and $c$
  there is a number that has $1$ in the $k$th position.
  Note that without loss of generality
  $a = (p_\ell, \dots, p_{k + 1}, 1, p_{k - 1}, \dots, p_0)_2$.
  Consider $a' = (p_\ell, \dots, p_{k + 1}, 0,
    p_{k - 1} \lxor r_{k - 1}, \dots, r_0 \lxor p_0)_2$. It is clear that
  $a' < a$ and $a' \bitwisexor b \bitwisexor c = 0$.
  Hence, $(a', b, c)$ is a P-position and therefore, $(a, b, c)$ is an
  N-position.

  Finally, let us consider $(a, b, c)$ such that
  $a \bitwisexor b \bitwisexor c = 0$. Assume that there is a move to a
  position $(a', b, c)$ such that $a' \bitwisexor b \bitwisexor c = 0$.
  This implies that
  $(a' \bitwisexor b \bitwisexor c) \bitwisexor
      (a \bitwisexor b \bitwisexor c) =  a \bitwisexor a' = 0$, whence $a = a'$.
\end{proof}

\begin{exercise}
  Prove that a position $(a_1, \dots, a_n)$ in $k$-pile Nim is a P-position
  iff $a_1 \bitwisexor \dots \bitwisexor a_k = 0$.
\end{exercise}


Using Bouton's theorem we can analyse games that do not look like the game of
Nim. An important example of such a game is the \emph{turning turtles} game.
\begin{game}[Turning Turtles Game]
  Given a horizontal line of $m$ coins with some coins showing heads and some
  tails. Each turn, a player have to flip one coin from head to tail, and in the
  same time (if he/she wants), flip one more coin to the left of it.
\end{game}
The game is equivalent to the game of Nim with each coin showing head in $k$th
position equals to a pile of $k$ chips.
\begin{enumerate}
  \item Flip one coin of position $k$ from head to tail. This move is equivalent
    with removing all stones from a pile with $k$ stones.
  \item Flip one coin of position $k$ from head to tail and flip another coin
    (from tail to head) to the left of it in position $t$. This move is equivalent
    with removing some stones from a pile with $k$ stones leaving $t$ stones in
    that pile.
  \item Flip one coin of position $k$ from head to tail and flip another coin 
    (from head to tail) to the left of it in position $t$. This move is equivalent
    with removing some stones from a pile with $k$ stones leaving $t$ stones in that
    pile. Note that having two piles with a same number of stones is the same as
    having none of both piles by Bouton's theorem and the fact that 
    $x \bitwisexor x = 0$.
\end{enumerate}

\section{Error-correcting Codes}
Using developed methods we may solve the following question about an extension
of the game discussed in \Cref{chapter:strong-induction}:
Alice has chosen a number from $1$ to $1000$. Bob wants to
guess the number so he is asking Alice ``yes'' or ``no'' questions.
How many questions does Bob need to ask to determine the number in the
worst-case scenario if Alice may lie to one of Bob's questions?

It is clear that Bob can use the same strategy as if Alice cannot lie bu ask
every question twice and if the answer for the same question are different ask
it third time. Hence, the number of questions necessary is at most $21$. In this
section we show how Bob can guess the number using only $15$ questions.

To solve this question we need to introduce the notion of error-correcting
codes. To illustrate this notion consider completely different scenario.
Two people wish to send a number from $\range{n}$ by sending $m$ bits;
unfortunately the channel connecting them is not unreliable so among $m$ sent
bits $d$ may be corrupted, but they still want to be able to reconstruct the
original message. How can they do this? To answer this question we need to
introduce error-correcting codes.
\begin{definition}
  Let $x, y \in \set{0, 1}^m$ we say that the Hamming distance 
  $\hammingDist{x}{y}$between $x$ and
  $y$ is the number of positions where these two strings are different.

  A function $C : \range{n} \to \set{0, 1}^m$ is an error-correcting code
  correcting $d$ errors if for any $z \in \set{0, 1}^m$ there is at most one 
  $i$ such that $\hammingDist{C(i)}{z} \le d$.
\end{definition}
It is clear that if they know such a code, they can send a number from
$\range{n}$ and reconstruct it back. We may also note that if Bob knows such a
code for $n = 1000$ and $d = 1$, then Bob may guess the Alice's number using $m$
questions.


Hence, to prove that it is enough for Bob to ask 15 questions, we need to show
that there is an error-correcting code with $n = 1000$, $m = 15$, and $d = 1$.
One may notice that instead of constructing an error correcting code, it is
enough to construct a set $S \subseteq \set{0, 1}^m$ such that $S$ has $n$
elements and $\hammingDist{x}{y} \ge 2d + 1$ for any $x, y \in S$. Indeed, assume
such a set exists; without loss of generality $S = \set{x_1, \dots, x_n}$.
Consider $C : \range{n} \to \set{0, 1}^m$ such that $C(i) = a_i$. We claim that
$C$ is an error-correcting code correcting $d$ errors. Assume the opposite;
i.e., that for some $z$ there are $i$, $j$ such that 
$\hammingDist{C(i)}{z}, \hammingDist{C(j)}{z} \le d$. This implies that 
$\hammingDist{C(i)}{C(j)} = \hammingDist{a_i}{a_j} \le 2d + 1$ which is a
contradiction.


In the rest of the section we construct such a set $S$ for $n \ge 1000$, $m =
15$, and $d = 1$. Note that any two P-positions in turning turtles are different
in at least $3$ positions. Hence, let us consider the set 
$S \subseteq \set{0, 1}^m$ of P-positions in turning turtles with $m$ coins
(heads are represented by $0$ and tails are represented by $1$). To finish the
construction of the set, we use the following lemma that we prove in
\Cref{chapter:principles}.
\begin{lemma}
\label{lemma:turning-turtles-number-P}
  There are at least $2^{2^r - r - 1}$ P-positions in turning turtles with 
  $2^r - 1$ coins.
\end{lemma}

Combining this together we prove the following theorem.
\begin{theorem}
  Let $r$ be an integer, let $n = 2^{2^r - r - 1}$, and $m = 2^r - 1$.
  There is an error-correcting code $C : \range{n} \to \set{0, 1}^m$
  correcting $1$ error.\footnote{%
    Nonetheless that these constructions seem very artificial and theoretical,
    in fact a code based on a very similar idea, the binary Golay code, was used
    to encode pictures of Jupiter and Saturn sent by Voyager~1 and Voyager~2.
  }
\end{theorem}
The code constructed in this section is called Hamming code and in
\Cref{chapter:binomials} we will show that it is optimal; i.e., we cannot reduce
$m$ without reducing $n$.

\begin{chapterendexercises}
  \exercise \emph{Nimble} is a game played on a board made of a line of squares
    labeled $0$, $1$, \dots A finite number of coins is placed on squares with
    possibly more than one coin on a square. A move consists in taking one coin
    and moving it to any square to the left, possibly over some coins and
    possibly onto a square containing some coins. Players, as usual, alternate
    moves and the game ends when all the coins are on the square $0$. The last
    player to move wins. Determine P- and N-positions in this game.
  \exercise In \emph{staircase Nim}, a staircase of $n$ steps contains coins of
    some of the steps. We can describe any position by a tuple $(a_1, \dots, a_n)$,
    where $a_i$ is the number of coins on the $i$th step. A move in staircase
    Nim consists of moving any number of coins from the $i$th step to 
    $(i - 1)$th step. Coins reaching the ground (step $0$) are removed from the
    play. The game ends when all the coins are removed. Players alternate moves;
    the last player to move wins. Show that $(a_1, \dots, a_n)$ is a P-position
    if coins on odd steps $(x_1, x_3, \dots)$ form a P-position in the game of
    Nim.
  \exercise In \emph{index-$k$ Nim} players can remove chips from at least
    one but up to $k$ different piles. Show that $(a_1, \dots, a_n)$ is a
    P-position in index-$k$ Nim iff 
    $\sum_{j = 1}^n b_{j, i} \equiv 0 \pmod{k + 1}$ for every 
    $i \in \set{0, 1, \dots, \ell}$, where $(b_{j, \ell} \dots b_{j, 0})_2$ is
    the binary representation of $a_j$.
\end{chapterendexercises}
