\chapter{The Pigeonhole Principle}
The principle we are going to discuss in this chapter is very simple: it states
that if you have more objects than boxes, then you cannot put all the objects to
boxes without puting two objects in the same box.

More formally the principle can be formulated as follows: if $n > m$, then any
function from $[n]$ to $[m]$ is not an injection. This simple statement is
famous in mathematics and called \textit{the pigeonhole principle}\footnote{%
  The pigeonhole principle is also called the Dirichlet principle, after the
  German mathematician G. Lejeune Dirichlet, who demonstrated that there were
  at least two Parisians with the same number of hairs on their heads.
}.

\begin{theorem}
  Let $X$ and $Y$ be some sets such that $|X| > |Y|$. Then for any function
  $f : |X| \to |Y|$ there are $x_0 \neq x_1 \in X$ such that $f(x_0) = f(x_1)$.
\end{theorem}
\begin{proof}
  The statement follows from
  \Cref{theorem:injections-surjections-inequalities}.
\end{proof}

This simple statement is very handy in combinatorics. For example, using this
statement one may prove that in any group of more than $12$ people there are
two people who were born in the same month.

Assume that there are $n$ people in the group and $n > 12$.
Consider the following function $f : [n] \to [12]$ such that $f(i) = j$ if the
$i$th person was born in $j$th month. Note that $f$ is not an injection since
$n > 12$ i.e. there are $i_0 \neq i_1$ such that $i_0$th and $i_1$th person are
born in the same month.

We may also prove that in any group of people there are two people who are
friends with the same number of people in the group.

Assume the number of people is $n$. It is easy to see that every person may
have at most $n - 1$ friends. Hence, we may define a function $f: [n] \to
\set{0, \dots, n - 1}$ such that $f(i)$ is equal to the number of friends in
this group of the $i$th person in this group.
We need to consider two cases.
\begin{itemize}
  \item If $\Im f \subseteq [n - 1]$. In this case
    $|[n]| > |\Im f|$ and $f$ is not an injection.
  \item Otherwise, note that it is not possible that $(n - 1) \in \Im f$
    since it there is a friend of nobody it is not possible that there is a
    friend of everyone. Hence, $f : [n] \to \set{0, 1, \dots, n - 2}$ and $f$ is
    not an injection.
\end{itemize}

\section{The Generalized Pigeonhole Principle}
One may generalize the pigeonhole principle in the following way.
If $N$ objects are placed into $k$ boxes, then there is at least one box
containing at least $\ceil{N / k}$ objects.
\begin{theorem}
\label{theorem:generalized-pigeonhole-principle}
  Let $X$ and $Y$ be some sets. Then for any function $f : |X| \to |Y|$ there
  are $x_1, \dots, x_\ell \in X$ such that
  \begin{itemize}
    \item $f(x_i) = f(x_j)$,
    \item $x_i \neq x_j$ for any $i \neq j \in [\ell]$, and
    \item $\ell \ge \ceil{|X| / |Y|}$
  \end{itemize}
\end{theorem}

\begin{exercise}
    Prove \Cref{theorem:generalized-pigeonhole-principle}.
\end{exercise}

Using this theorem we can prove that if we draw $9$ cards out of a deck of
cards, we are guaranteed that at least three of them are of the same suit.
Indeed, there are $4$ suits and by pigeonhole principle if we put each card to
one out of four boxes according to their suit, one of the boxes should have
at least $\ceil{9 / 4} = 3$ cards.

\section*{End of The Chapter Exercises}
\begin{exercises}
  \exerciseitem Prove that for every integers $a_1$, \dots, $a_n$ there are
    $k > 0$ and $\ell \ge 0$ such that $k + \ell \le n$ and
    $\sum\limits_{i = k}^{k + \ell} a_i$ is divisible by $n$.
  \exerciseitem Let $S \subseteq [20]$ be a set. Show that if
    $|S| \ge 13$, then there are $a, b \in S$ such that $a - b = 6$.
  \exerciseitem Sasha is training for a triathlon. Over a $30$ day period, he
    pledges to train at least once per day, and $45$ times in all. Then there
    will be a period of consecutive days where he trains exactly $14$ times.
  \exerciseitem Show that among any n + 1 positive integers not exceeding 2n
    there must be an integer that divides one of the other integers.
\end{exercises}
