\chapter{Permutations and Binomial Coefficients}
\section{Counting Functions}
Assume we have two finite sets $X$ and $Y$. The first question we may ask is:
how many function exist from $X$ to $Y$.


\begin{theorem}
\label{theorem:number-of-functions}
  Let $X$ and $Y$ be some finite sets. We denote by $Y^X$ the set of all
  functions from $X$ to $Y$. Then $|Y^X| = |Y|^{|X|}$.
\end{theorem}
\begin{proof}
  For simplicity we prove the statement in the case when $X = [n]$.
  Proof of this statement is similar to \Cref{theorem:cardinality-of-power-set}.
  First, we prove that $|Y^X| = |Y^n|$. In order to do this we construct
  a bijection from $Y^n$ to $Y^X$. Consider the function
  \[
    F(y_1, \dots, y_n) = f_{y_1, \dots, y_n},
  \]
  where $f_{y_1, \dots, y_n}(i) = y_{i}$. We need to show that it is indeed a
  bijection.
  \begin{description}
    \item[Injection:] assume $F$ is not an injection, i.e. there are
      $(a_1, \dots, a_n) \neq (b_1, \dots, b_n) \in Y^{|X|}$ such that
      $F(a_1, \dots, a_n) = F(b_1, \dots, b_n)$. Note that this implies that
      $a_i = f_{a_1, \dots, a_n}(i) =
        f_{b_1, \dots, b_n}(i) = b_i$ for all $i \in [n]$. Thus
      $(a_1, \dots, a_n) = (b_1, \dots, b_n)$, which is a contradiction.
    \item[Surjection:] consider some function $f \in Y^X$. Note that
      $F(f(1), \dots f(n)) = f$. Thus $f \in \Im F$.
  \end{description}

  Finally, we may notice that, by the multiplicative principle, $|Y^{|X|}| =
  |Y|^{|X|}$.
\end{proof}

\begin{exercise}
  Finish the proof of \Cref{theorem:number-of-functions} by proving that
  the statement holds for any set $X$.
\end{exercise}

However, what if we need to find size of a subset of $Y^X$ satisfying some
constraint. For example, we may try to find size of the set
\[
  (Y)_X = \set[f\text{ is an injection}]{f \in Y^X}.
\]
First, let us try to do this informally. Assume that $X = [n]$ and $|Y| = m$,
to define $f \in (Y)_X$ we need to choose images of $1$, $2$, \dots, $n$. There
are $m$ possible ways to select an image of $1$, $m - 1$ ways to define $f(2)$
since we can not use the value selected for $1$ etc. Hence,
$|(Y)_X| = m (m - 1) \dots (m - n + 1)$ (we denote this number as $(m)_n$).


\begin{theorem}
\label{theorem:number-of-injections}
  Let $X$ and $Y$ be some sets. Then $|(Y)_X| = (|Y|)_{|X|}$.
\end{theorem}
\begin{proof}
  Let us prove this statement for $X = [n]$. We prove this using induction by
  $n$. The base case, for $n = 1$, is clear. Now we need to prove the induction
  step from $n$ to $n + 1$. By the induction hypothesis, for any $m$, the
  number of injections from $[n]$ to $Y$ is equal to $(|Y|)_n$.

  Fix some $m$ and some set $Y$ of cardinality $m$. Note that
  \[
    |(Y)_X| =
    |\set[v \not\in \Im f]{(f, v) \in (Y)_{[n - 1]} \times [m]}|.
  \]
  It is easy to see that $|\set[v \not\in \Im f]{(f, v)}| = m - n + 1$
  for any $f \in (Y)_{[n - 1]}$ and
  \[
    \set[v \not\in \Im f]{(f, v) \in (Y)_{[n - 1]} \times [m]} =
    \bigcup\limits_{f \in (Y)_{[n - 1]}} \set[v \not\in \Im f]{(f, v)}.
  \]
  As a result, $|(Y)_X| = (m)_{n - 1} \cdot (m - n + 1) = (m)_n$.
\end{proof}

The special case of this result is that there are $n (n - 1) \dots 1$ different
bijections from $[n]$ to $[n]$. Such bijections are called permutations and the
number is denoted by $n!$.

\begin{exercise}
  Finish the proof of \Cref{theorem:number-of-injections} by proving that
  the statement holds for any set $X$.
\end{exercise}

\section{Counting Subsets}
Recall that we denoted the set of all subsets of $X$ by $2^X$. The reason for
this notation is that $|2^X| = 2^{|X|}$.

A quite famous example of a subset of this set is the set
\[
  \binom{X}{k} = \set[|X| = k]{A \subseteq [n]}.
\]
In other words, it is the set of all possible ways to select $k$ elements from
$X$. Size of this set we denote by $\binom{|X|}{k}$ and call it a binomial
coefficient.

Let us find several useful properties of the binomial coefficients.
\begin{theorem}
  Let $n$, $m$, and $k$ be some integers such that $m > n > k$.
  \begin{itemize}
    \item $\binom{n}{k} = \binom{n}{n - k}$.
    \item $\sum\limits_{k = 0}^n \binom{n}{k} x^k y^{n - k} = (x + y)^n$.
    \item $\binom{n}{k} = \binom{n - 1}{k - 1} + \binom{n - 1}{k}$.
    \item $\binom{n}{k} = \frac{(m)_n}{n!}$.
  \end{itemize}
\end{theorem}
