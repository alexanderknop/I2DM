\chapter{Uncountable Sets}

In the previous chapter we proved that many sets are countable; moreover, we
proved that standard operations such as union, intersection, and product of
countable sets is also countable. This chapter shows that we know at least one
uncountable set. Moreover, this chapter gives tools to compare cardinalities of
sets.

\section{Cardinality of Reals}
\begin{theorem}[Cantor]
\label{theorem:reals-are-uncountable}
  The set of real number $\R$ is uncountable.
\end{theorem}

To prove this theorem we need the following two lemmas.
\begin{lemma}
\label{lemma:segment-is-uncountable}
  The set $\set[0 < x < 1]{x \in \R}$ of real numbers between $0$ and $1$ is
  uncountable.
\end{lemma}

\begin{lemma}
\label{lemma:segment-equipotent-to-reals}
  The set $\R$ is equipotent to $\set[0 < x < 1]{x \in \R}$.
\end{lemma}

\begin{proof}[Proof of \Cref{theorem:reals-are-uncountable}]
  Assume that $\R$ is denumerable. Then by
  \Cref{remark:denumerable-equipotent-to-denumerable,lemma:segment-equipotent-to-reals},
  $\set[0 \le x \le 1]{x \in \R}$ is also denumerable, which contradicts
  \Cref{lemma:segment-is-uncountable}.
\end{proof}

To complete the proof of \Cref{theorem:reals-are-uncountable} we need to prove
these two lemmas.
\begin{proof}[Proof of \Cref{lemma:segment-is-uncountable}]
  Assume $\set[0 < x < 1]{x \in \R}$ is countable; this implies that 
  $S = \set[0 \le x \le 1]{x \in \R}$ is also countable. 

  Note that each number $x \in S$ can be represented as an infinite decimal 
  \[
    x = 0.a_1 a_2 a_3 \dots
  \]
  where $0 \le a_i \le 9$ for all $i \in \N$ (the number $1$ can be represented
  as $0.\dot{9}$).

  Let $f : \N \to S$ be a bijection, and let $f(n) = .a_{n, 1} a_{n, 2} \dots$.
  Consider the sequence $b_1$, $b_2$, \dots such that 
  \[
    b_i = 
    \begin{cases}
      0 & \text{if } a_{i, i} \neq 0 \\
      1 & \text{otherwise}
    \end{cases}.
  \]
  Let $y = 0.b_1 b_2 \dots$ and let $m \in \N$ be an integer such that $f(m) =
  y$. Note that 
  \[
    .a_{m, 1} a_{m, 2} \dots = f(m) =  0.b_1 b_2 \dots
  \]
  However, $a_{m, m} \neq b_m$ which is a contradiction.
\end{proof}

\begin{proof}[Proof of \Cref{lemma:segment-equipotent-to-reals}]
  To prove this lemma we start from proving that $\R$ is equipotent to 
  $\set[-\pi / 2 < x < \pi / 2]{x \in \R}$; indeed, we may consider the tangent
  function which is a bijection between these two sets.

  The next step is to note that the sets $\set[-\pi / 2 < x < \pi / 2]{x \in \R}$ and 
  $\set[0 < x < 1]{x \in \R}$ are also equipotent. Consider the function 
  $f : \set[0 < x < 1]{x \in \R} \to \set[-\pi / 2 < x < \pi / 2]{x \in \R}$
  such that $f(x) = \pi (x - 1 / 2)$. It is clear that $f$ is a bijection, which
  finishes the proof.
\end{proof}

\begin{exercise}
  Show that $(\set[0 < x < 1]{x \in \R})^2$ and $\set[0 < x < 1]{x \in \R}$ are equipotent.
\end{exercise}

\section{Inequalities Between Cardinalities}
Since $\N \subseteq \R$ the statement that $\R$ is uncountable can be
interpreted as `the set of reals is larger than the set of integers (or the set
of rationals)'. The following definition allows to make this reformulation
precise.
\begin{definition}
  Let $X$ and $Y$ be two sets. 
  \begin{itemize}
    \item We say that $X$ and $Y$ have the same cardinality if they are
      equipotent; i.e., if $|X| = |Y|$.
    \item We write that $|X| \ge |Y|$ and say that the cardinality of $X$ is at
      most the cardinality of $Y$ if there is an injection from $X$ to $Y$.
    \item We also write $|X| < |Y|$ and say that the cardinality of $X$ is less
      than the cardinality of $Y$ if $|X| \neq |Y|$ and $X \le |Y|$.
  \end{itemize}
\end{definition}
Cantor's theorem says that $|\R| > |\N|$ or $|\R| > \aleph_0$.\footnote{%
  A quesiotn that immidiately come in mind after lookin at this reformulation is
  whether exists something in between; i.e., whether exists a set $S$ such that
  $\aleph_0 < |S| < |\R|$. The hypothesis that such a set exists is called
  \emph{the continuum hypothesis}. In 1963, Paul Cohen proved that it is not
  possible to prove or disprove thus hypothesis starting from axioms of the set
  theory.
}

Moreover, Cantor proved that there are even bigger sets
\begin{theorem}
  For any set $X$, $|2^X| > |X|$.
\end{theorem}
\begin{proof}
  The idea of this proof is simialr to the proof of Cantor's theorem
  (\Cref{theorem:reals-are-uncountable}). 
  
  Firs of all we may notice that $|2^X| \ge |X|$ since $f : X \to 2^X$ such that
  $f(x) = \set{x}$ is an injection.
  
  It is also clear that the set $2^X$ is equipotent to the set $\set{0, 1}^X$
  (this can be proven similarly to \Cref{corollary:power-set-and-set-of-binary-strings}).

  Let us now prove that $|\set{0, 1}^X| \neq |X|$. Assume this is not true and
  there is a bijection $F : X \to \set{0, 1}^X$. Let $f_x = F(x)$.
  Consider $g : X \to \set{0, 1}$ such that 
  \[
    g(x) = 
    \begin{cases}
      0 & \text{if } f_x(x) = 1 \\
      1 & \text{if } f_x(x) = 0 \\
    \end{cases}
  \]
  (note that for any $x$, $g(x) \neq f_x(x)$). It is clear that
  $g \in \set{0, 1}^X$; however, if $g = F(y) = f_y$, then $g(y) = f_y(y) \neq
  f_y(y)$, which is a contradiction. Therefore $F$ is not a bijection.
\end{proof}

We devoted the whole chapter to the pigeonhole principle, unsurprisengly one may
prove a similar theorem for infinite sets.\footnote{%
  This theorem is often called Schr\"{o}der--Bernstein theorem since Schr\"{o}der
  and Bernstein published independently proofs of this theorem in 1898. Cantor
  is often added because he first stated the result in 1887, while
  Schr\"{o}der's name is often omitted because his proof turned out to be
  flawed. Howerver, the name of Dedekind, who first proved it is not mentioned
  at all.
}
\begin{theorem}[Schr\"{o}der--Bernstein]
  Let $X$ and $Y$ be two nonempty sets such that $|X| > |Y|$. Then any function
  $f : X \to Y$ is not an injection; i.e., there are $x_1, x_2 \in X$ such that
  $f(x_1) = f(x_2)$.
\end{theorem}


We are not going to prove this theorem; however, let us formulate a very
important corollary of this theorem.
\begin{corollary}
\label{corollary:schroder-bernstein}
  Let $X$ and $Y$ be two nonempty sets. If $|X| \le |Y|$ and $|Y| \ge |X|$, then
  $|X| = |Y|$.
\end{corollary}

Using this theorem we may prove that 
\[
  \left|
    \set[x, y \in \R, x^2 + y^2 = 1]{(x, y)}
  \right| = |\set[-1 \le x \le 1]{x \in \R}|.
\]
First of all note that 
\[
  \set[x, y \in \R, x^2 + y^2 = 1]{(x, y)}
  \subseteq \set[-1 \le x \le 1]{x \in \R};
\]
hence
\[
  \left|
    \set[x, y \in \R, x^2 + y^2 = 1]{(x, y)}
  \right| \le |\set[-1 \le x \le 1]{x \in \R}|.
\]
In addition, one may prove using scaling and shift that
\[
  |\set[-1 \le x \le 1]{x \in \R}| = 
  \left|
    \set[-\frac{1}{\sqrt{2}} \le x \le \frac{1}{\sqrt{2}}]{x \in \R}
  \right|.
\]
Finally, 
\[
  \set[-\frac{1}{\sqrt{2}} \le x \le \frac{1}{\sqrt{2}}]{x \in \R}
  \subseteq 
  \set[x, y \in \R, x^2 + y^2 = 1]{(x, y)}
\]
hence
\[
  \left|
    \set[-\frac{1}{\sqrt{2}} \le x \le \frac{1}{\sqrt{2}}]{x \in \R}
  \right|
  \le
  \left|
    \set[x, y \in \R, x^2 + y^2 = 1]{(x, y)}
  \right|.
\]
Combining all this together we can obtain that
\begin{gather*}
  \left|
    \set[x, y \in \R, x^2 + y^2 = 1]{(x, y)}
  \right| \le |\set[-1 \le x \le 1]{x \in \R}| \\
  \text{and}\\
  |\set[-1 \le x \le 1]{x \in \R}| \le
  \left|
    \set[x, y \in \R, x^2 + y^2 = 1]{(x, y)}
  \right|.
\end{gather*}
Hence, using the Schr\"{o}der–Bernstein theorem we can prove the equality.

\begin{chapterendexercises}
    \exercise[recommended] Prove that $|\set{0, 1}^X| = |2^X|$.
    \exercise[recommended] Prove \Cref{corollary:schroder-bernstein}
    \exercise Show that $\set{0, 1}^\N$ is equipotent to $\set{0, 1, 2}^\N$.
    \exercise Show that if a set $A \subseteq \R^2$ contains a line, then $A$
      and $\R$ are equipotent.
\end{chapterendexercises}
