\chapter{Universal Functions}
It is known that we may write a program that gets another program as an argument
and run it. (Such programs are known as interpreters.) To use this observation
we give the following definition and theorem.
\begin{definition}
  We say that a function $U$ is \emph{universal function} (for the set of univariate
  computable functions) iff for each $n \in \N$,
  \[
    U_n : x \mapsto U(n, x)
  \]
  (we say that $U_n$ is a section of $U$)
  is computable and any univariate computable function is among $U_n$'s.
\end{definition}

\begin{theorem}
\label{theorem:universal-function-computable}
  There is a computable universal function $U$.
\end{theorem}

\begin{exercise}
  Assume that every section of a function $U$ is computable.
  Is it necessary true that $U$ is computable?
\end{exercise}

Similarly to the notion of universal function we may define the notion
of universal sets.
\begin{definition}
  Let $F \subseteq 2^\N$. We say that $W \subseteq \N^2$ is universal for $F$
  if $F = \set[n \in \N]{\set[(n, x) \in F]{x}}$.
\end{definition}

\begin{theorem}
  There is a enumeratable set $W$ such that it is universal for the set of
  all enumeratable subsets of $\N$.
\end{theorem}
%
% \section{Enumeratable but Notdecidable Set}
%
% \section{Total Functions}
% \Cref{theorem:universal-function-computable} says that there is a universal
% function for the set of computable functions.
