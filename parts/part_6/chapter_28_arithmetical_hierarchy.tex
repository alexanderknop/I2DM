\chapter{Arithmetical Hierarchy}

\Cref{theorem:enumerable-via-computable-funcitons} proves that any decidable set
is the image of some total computable function from $\N$. Identifying sets and
predicates we can prove the following theorem.
\begin{theorem}
  A predicate$A(x)$ of natural numbers is enumerable iff there is a decidable
  predicate $B(x, y)$ such that the following formula is always true:
  \[
    A(x) \iff \exists y \  B(x, y).
  \]
  (Here and in the sequel, we are going to write $\exists x \  P(x)$ denotes the
  statement ``there is an $x$ from the domain of $P$ such that $P(x)$ is true.)
\end{theorem}

A natural questions can be asked: ``What can be said about other combinations of
quantifiers?''.

It is easy to see that if $A(x)$ is a predicate such that 
\[
  A(x) \iff \exists y \  \exists z \  C(x, y, z),
\]
where $C(x, y, z)$ is decidable, then $A(x)$ is enumerable.
Indeed, let $B(x, \pair{y}{z}) = C(x, y, z)$; then $B(x, w)$ is decidable and 
\[
  A(x) \iff \exists y \  \exists z \  B(x, \pair{y}{z}) \iff 
    \exists w \  B(x, w).
\]
If $A(x)$ is a predicate such that 
\[
  A(x) \iff \forall y \  B(x, y),
\]
where $B(x, y)$ is decidable, then the negation of $A(x)$ is enumerable (we call
such sets and predicates \emph{coenumerable}). Indeed, 
\[
  \lnot A(x) \iff \exists y \  \lnot B(x, y),
\]
and $\lnot B(x, y)$ is decidable since the compliment of a decidable set is
decidable.

This question gives rise to the following definition.
\begin{definition}
  We say that a predicate $A$ on $\N$ belongs to the class $\Sigma_n$ iff there
  is a deciable predicate $B$ on $\N^{n + 1}$ such that 
  \[
    A(x) \iff 
    \exists y_1 \  \forall y_2 \  \exists y_3 \  \dots B(x, y_1, \dots, y_n).
  \]
  Similarly, a predicate $A$ on $\N$ belongs to the class $\Sigma_n$ iff there
  is a deciable predicate $B$ on $\N^{n + 1}$ such that 
  \[
    A(x) \iff 
    \forall y_1 \  \exists y_2 \  \forall y_3 \  \dots B(x, y_1, \dots, y_n).
  \]
\end{definition}

Previous observations can be generalized as follows.
\begin{theorem}
  \begin{enumerate}
    \item The set $\Sigma_1$ consists of enumerable sets.
    \item If a predicate belongs to $\Sigma_n$, then its negation belongs to
      $\Pi_n$.
    \item The set $\Sigma_n$ does not change if we allow groups of quantifiers
      of the same type instead of single quantifiers.
  \end{enumerate}
\end{theorem}

Like enumerable and decidable sets, $\Sigma_n$ and $\Pi_n$ sets have several
good properties.
\begin{theorem}
  Union and intersection of two $\Sigma_n$ ($\Pi_n$) sets are also $\Sigma_n$
  ($\Pi_n$) sets.
\end{theorem}
